%!TEX TS-program = pdflatex
%!TEX root = tesi.tex
%!TEX encoding = UTF-8 Unicode


\section{Introduzione}

Il \TeX{} fu pensato ai lontani tempi in cui i terminali
grafici erano un lusso stravagante e l'unico modo di
comunicare coi calcolatori era di battere tasti della
tastiera. Ancora oggi chi scrive in \TeX{} deve inserire i
comandi di formattazione mescolati al testo ed è meglio
se ha un manuale a portata di mano. Per rialzare
il morale basta che si rammenti che il \TeX{} è ancora
solidamente il più sofisticato sistema di impaginazione
per testi scientifici, e che è gratis.

\subsection{Prima sezione}

Per cominciare a scrivere in \TeX{} bisogna individuare i
\emph{caratteri speciali}, o \emph{caratteri di
controllo}\index{caratteri di controllo}, che servono a
distinguere il testo dai comandi. Chi lavora con la
tastiera italiana\index{tastiera italiana} dovrà faticare
all'inizio, perché alcuni di quei caratteri importanti
si raggiungono solo attraverso certe combinazioni di
tasti, che per di più cambiano da un sistema
operativo\index{sistema operativo} a un altro. Consultate
il manuale del vostro calcolatore o chiedete aiuto agli
esperti se non vi raccapezzate.
